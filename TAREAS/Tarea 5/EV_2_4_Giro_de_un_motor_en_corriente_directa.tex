\documentclass[11pt,a4paper]{article}
\usepackage{graphicx}
\usepackage{amsmath}
\usepackage{amssymb}
\usepackage{mathrsfs}
\usepackage{cancel}


\begin{document}
\begin{center}
\textbf{Tarea 5}\\
EV 2-4 Giro de un motor en corriente directa
\end{center}

\begin{center}
Cedano Gutierrez Nancy Noemi\\
15-Octubre-2019\\
Universidad Politecnica de La Zona Metropolitana de Guadalajara
\end{center}
\section{Principio del giro de los motores}
El principio de funcionamiento dice que la repulsion que ejercen los polos magneticos de un iman permanente cuando, de acuerdo a la ley de lorentz, interactuan con los polos magneticos de un electroiman que se encuentra montado sobre un eje. 	Este electroiman se le conoce como rotor y su eje le permite girar libremente entro los polos magneticos norte y sur del iman permanente situado dentro de la carcasa o cuerpo del motor.\\
Cuando la corriente eléctrica circula por la bobina de este electroimán giratorio, el campo electromagnético que se genera interactúa con el campo magnético del imán permanente. Si los polos del imán permanente y del electroimán giratorio coinciden, se produce un rechazo y un torque magnético o par de fuerza que provoca que el rotor rompa la inercia y comience a girar sobre su eje en el mismo sentido de las manecillas del reloj en unos casos, o en sentido contrario, de acuerdo con la forma que se encuentre conectada al circuito la pila o la batería.\\

\section{Funcionamiento}
En el motor de corriente directa el conmutador sirve para conmutar o cambiar constantemente El sentido de circulación de la corriente eléctrica a través del enrollado de la bobina del rotor  cada  vez que completa media vuelta De esa forma el polo norte del electroimán coincidirá siempre con el también polo norte del imán permanente y el polo sur con el polo sur del propio imán. Al coincidir siempre dos polos magnéticos, que en todo momento van a ser iguales, se produce un rechazo constante entre ambos, lo que permite al rotor mantenerse girando ininterrumpidamente sobre su eje durante todo el tiempo que se encuentre conectado a la corriente eléctrica.\\
Como resultado, cuando en el electroimán se forma el polo norte, de inmediato el también polo norte del imán permanente lo rechaza. Al mismo tiempo el polo sur que se forma en el extremo opuesto, es rechazado igualmente por el polo sur del propio imán; por tanto se produce una fuerza de repulsión en ambos extremos del rotor al enfrentarse y coincidir con dos polos iguales en el imán permanente. Si bajo esas condiciones aplicamos la “Regla de la mano izquierda” y tomamos como referencia, por ejemplo, la parte de la bobina donde se ha formado el polo norte en el electroimán, comprobaremos que al romper la inercia inicial, comenzará a girar en dirección contraria a las manecillas del reloj.\\


\section{Conclusion}
En pocas palabras, la función del conmutador es permitir el cambio constante de polaridad de la corriente en la bobina del electroimán del rotor para que sus polos cambien constantemente. Este cambio ocurre cada vez que el electroimán gira media vuelta y pasa por la zona neutra, momento en que sus polos cambian para que se pueda mantener el rechazo que proporciona el imán permanente. Esto permitirá que el electroimán del rotor se mantenga girando constantemente durante todo el tiempo que la batería o fuente de fuerza electromotriz (F.E.M.) se mantenga conectada al circuito del motor, suministrándole corriente eléctrica.


\end{document}