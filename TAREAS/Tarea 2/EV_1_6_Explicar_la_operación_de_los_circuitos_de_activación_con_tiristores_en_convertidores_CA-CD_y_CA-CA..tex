\documentclass[11pt,a4paper]{article}
\usepackage{graphicx}
\usepackage{amsmath}
\usepackage{amssymb}
\usepackage{mathrsfs}
\usepackage{cancel}

\begin{document}
\begin{center}
\textbf{TAREA 2}\\
\textbf{EV 1-6 Explicar la operacion de los circuitos de activacion con tiristores en convertidores CA-CD Y CA-CA}
\end{center}

\begin{center}
Cedano Gutierrez Nancy Noemi\\
\textbf{23-sep-2019}\\
\textbf{Universidad Politecnica de La Zona Metropolitana de Guadalajara}
\end{center}

\section{Introduccion}
El flujo de corriente para una carga puede ser controlado colocando un SRC en serie con la carga. La alimentacion de voltaje es comunmente una fuente de 60 Hz de ca, pero en ocasiones es cd para circuitos especiales\\
En la Universidad Politecnica de la Zona Metropolitana de Guadalajara, se realizaran diversas simulaciones para comprender un poco mas sobre los circuitos con tiristores en un conversor de corriente CA-CD Y CA-CA\\
Los diodos rectificadores proporcionan sólo un voltaje de salida fijo, sin embargo, para obtener voltajes de salida controlados, se usan tiristores con control de fase. 
El voltaje de salida de los rectificadores de tiristor se varía controlando el ángulo de retardo a, por lo tanto, el tiristor se activa aplicando un pulso corto a su compuerta y se desactiva por conmutación natural.
Su misión fundamental es proporcionar energía eléctrica en forma de corriente continua a partir de una fuente de corriente alterna. \\
\section{Objetivo}
explicar la operacion de los circuitos de activacion con tiristores.

\section{Rectificador de Media Onda:}
circuito rectificador de Media Onda con carga resistiva para un tiempo t = p, la corriente cae naturalmente a cero, causando que el SCR se desactive. Durante el ciclo negativo, el SCR bloquea el flujo de la corriente, y por lo tanto no hay voltaje aplicado a la carga resistiva. Posteriormente el SCR permanece inactivo hasta que se aplica una señal a la compuerta (Gate). El periodo de cero hasta la activación de la compuerta es conocido como ángulo de disparo (a). La relación del voltaje promedio de salida para este sistema está dado por la siguiente Ecuacion:
\begin{center}
Vo' = Vmax*(1+COS a) / 2*p\\
\end{center}
\section{circuito Rectificador de Media Onda con carga resistiva-inductiva :}
si el SCR es activado con cierto ángulo a de retraso, la corriente incrementa lentamente debido al inductor. El voltaje en la carga es positivo y el inductor almacena energía mediante un campo magnético. Sin embargo, cuando el semi-ciclo negativo desactiva al SCR, el campo magnético se descarga a través de la carga en sentido opuesto a su polaridad obteniendo un voltaje negativo, por lo tanto, el voltaje de salida promedio es menor que el voltaje promedio con una pura carga resistiva.
\section{Rectificador de Onda Completa Controlado:}
Basándose en las leyes de Kirchoff para las mallas creadas por los SCRs, se demuestra que ambas parejas no pueden conducir al mismo tiempo, así como también el voltaje
inverso que soporta cada pareja es el voltaje pico del generador. El promedio de salida de voltaje de corriente directa puede ser controlada desde cero hasta su máximo valor positivo variando el ángulo de disparo a, el cual se obtiene al sincronizar 2 disparos en una pareja de SCRs, esto queda representado por la siguiente ecuacion, la cual muestra la relación del voltaje promedio de salida.
\begin{center}
Vo' = Vmax*(1 + COS a) / p\\
\end{center}
\section{Rectificador de Onda Completa Semi-Controlado :}
Este circuito es la simplificación del puente rectificador de onda completa. Permite controlar ambos semiciclos de la onda usando solo 2 SCRs. se alimenta una carga de corriente directa mediante una fuente de corriente alterna.
Durante el semiciclo positivo, T1 está polarizado directamente cuando este se dispara en ?t= a, la carga se conecta a la alimentación de entrada a través de T1 y D2 durante el período a = ?t = p, por su parte, en el período p = t = (p+a), el voltaje de entrada es negativo y el diodo de libre circulación Dm tiene polarización directa, por lo que conduce para proporcionar la continuidad de corriente de la carga resistiva-inductiva Posteriormente la corriente de carga se transfiere de T1 y D2 a Dm, y ambos (T1 y D2) se desactivan. Durante el semiciclo negativo del voltaje de entrada, el SCR 2 queda con polarización directa y el disparo del T2 en ?t=p+a invierte la polarización de Dm. El diodo Dm se desactiva y la carga se conecta a la alimentación a través de T2 y D1, la alternancia en las activaciones de los pares Tiristor-Diodo.
\section{Rectificador Trifásico de Onda Completa Semi-Controlado:}
Para un periodo p/6 < ?t < 7p/6 el tiristor T1 se encuentra polarizado en directa, cuando
éste es disparado en ?t = p/6 + a hasta ?t = 7p/6, T1 y D1 conducen el voltaje
procedente de la línea de entrada y aparecerá directamente sobre la carga resistivainductiva.
Cuando se tiene ?t = p/6, el voltaje de entrada es negativo y el diodo Dm tiene
polarización directa, haciendo que entre en conducción y la corriente de carga fluya sobre
él, de este modo T1 y D1 pasan al estado de desactivación.
Estas consideraciones se realizan con todas las combinaciones posibles, dichas
combinaciones, tomando en cuenta las Leyes de Kirchoff muestran que solo puede
conducir un Tiristor a la vez en la mitad superior del puente (T1, T2 y T3), de igual forma
también se observa que sólo puede conducir un diodo a la vez en la mitad inferior del
puente (D1, D2 y D3); por lo tanto T1 y D1 no podrán conducir al mismo tiempo como
consecuencia de las anteriores observaciones, al igual que T2-D2 y T3-D3.
\section{Conclusiones}
Los tiristores han sido impresindibles en la mayoria de las aplicasiones industriales actuales. El espacion tan pequeño de uso y la fiabilidad de conversor que es este componente han aludido a su uso.
\end{document}