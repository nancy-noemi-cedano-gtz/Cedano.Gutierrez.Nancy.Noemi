\documentclass[11pt,a4paper]{article}
\usepackage{graphicx}
\usepackage{amsmath}
\usepackage{amssymb}
\usepackage{mathrsfs}
\usepackage{cancel}
\usepackage[utf8]{inputenc}
\usepackage[spanish]{babel}


\begin{document}
\begin{center}
\textbf{Tarea 5\\Calcular los Parametros de los Circuito de Transistores de Potencia}\

\end{center}

\begin{center}
Cedano Gutierrez Nancy Noemi\\
29-Octubre-2019\\
Universidad Politecnica de La Zona Metropolitana de Guadalajara
\end{center}
\section{TRANSISTOR DE POTENCIA}
El funcionamiento de los transistores de potencia es identico a un transistor normal, teniendo como caracteristica especial las altas tensiones e intensidaes  que tienen que soportar y las potencias a disparar.\\
\section{TIPOS DE TRANSISTORES DE POTENCIA}
Bipolar\\
Unipolar o FET\\
IGBT\\

\begin{tabular}{|p{50mm}|p{40mm}|p{40mm}|}
\hline
	PARAMETROS & MOS & BIPOLAR \\
\hline
	Impedancia de entrada & Alta (1010 ohmios) & Media (104 ohmios)\\
\hline
	Ganancia en corriente & Alta (107) & Media (10-100) \\
\hline
	Resistencia ON (saturacion)& Media/Alta & Baja \\
\hline
	Resistencia OFF ( corte) & Alta & Alta \\
\hline
	Voltaje aplicable & Alto (1000V) & Alto (1200V)\\
\hline
	Maxima temperatura de operacion & Alta (200  C) & Media (150 C)\\
\hline
	coste & Alto & Medio\\
\hline
	Frecuencia de trabajo & Alta(100-500 Khz) & Baja (10-80Khz)
	
\end{tabular}

\section{Funcionamiento}
Una limitacion de todos los dispositivos de potencia y en especial de los transistores bipolares, es que el paso de bloqueo a conduccion y viceversa no se hace instantaneamente, sino que siempre hay un retardo (ton,toff). Las causas fundamentales de estos retardos son las capacidades asociadas a las uniones colector- base y base- emisor y los tiempos de difusion y recombinacion de los portadores.\\

\section{CALCULOS DE POTENCIAS DISIPADAS EN CONMUTACION CON CARGA RESISTIVA}
Supongamos el momento origen en el comienzo del tiempo de subida (tr) de la corriente de colector. En estas condiciones (0 t tr) tendremos :
\begin{center}
Ic= Icmax * (t/tr)
\end{center}
donde IC más vale :
\begin{center}
Icmax= Vcc/R * (t/tr)
\end{center}
También tenemos que la tensión colector - emisor viene dada como:
\begin{center}
Vce= Vcc-R * ic
\end{center}
Sustituyendo, tendremos que :
\begin{center}
Vce(mayusculas)=Vcc -R * Vcc/R
\end{center}
Nosotros asumiremos que la VCE en saturación es despreciable en comparación con Vcc.
Así, la potencia instantánea por el transistor durante este intervalo viene dada por :
\begin{center}
p= Vce *ic = Vcc*1cmax*(t/tr)*(1-t/tr)
\end{center}
La energía, Wr, disipada en el transistor durante el tiempo de subida está dada por la integral de la potencia durante el intervalo del tiempo de caída, con el resultado:
\begin{center}
Wr=(Vcc*1cmax/4)*(2*tr/3)
\end{center}
De forma similar, la energía (Wf) disipada en el transistor durante el tiempo de caída, viene dado como:
\begin{center}
Wf=(Vcc*1cmax/4)*(2xtf/3)
\end{center}
La potencia media resultante dependerá de la frecuencia con que se efectúe la conmutación:
\begin{center}
Pav=F*(Wr+Wf)
\end{center}
Un último paso es considerar tr despreciable frente a tf, con lo que no cometeríamos un error apreciable si finalmente dejamos la potencia media, tras sustituir, como:
\begin{center}
Pc(av)=Vcc*1cmax/6*tf*f
\end{center}

\section{CALCULO DE POTENCIAS DISIPADAS EN CONMUTACION CON CARGA INDUCTIVA}
Arriba podemos ver la gráfica de la iC(t), VCE(t) y p(t) para carga inductiva. La energía perdida durante en ton viene dada por la ecuación:
\begin{center}
Wtan=1/2*V*1c(sat)*(t1+t2)
\end{center}
Durante el tiempo de conducción (t5) la energía perdida es despreciable, puesto que VCE es de un valor ínfimo durante este tramo.
Durante el toff, la energía de pérdidas en el transistor vendrá dada por la ecuación:
\begin{center}
Wtoff= 1/2*V*1c(sat)*(t3+t4)
\end{center}
La potencia media de pérdidas durante la conmutación será por tanto:
\begin{center}
Ptot(av)=Wton+Wtoff/T=F*(Wton+Wtoff)
\end{center}
Si lo que queremos es la potencia media total disipada por el transistor en todo el periodo debemos multiplicar la frecuencia con la sumatoria de pérdidas a lo largo del periodo (conmutación + conducción). La energía de pérdidas en conducción viene como:
\begin{center}
Wcond=Vc(sot)*Ic(sat)*ts
\end{center}


\end{document}