\documentclass[11pt,a4paper]{article}
\usepackage{graphicx}
\usepackage{amsmath}
\usepackage{amssymb}
\usepackage{mathrsfs}
\usepackage{cancel}

\begin{document}
\begin{center}
\textbf{TAREA 3}\\
\textbf{EV 2-2 Explicar Los arreglos y Parametros de los amplificadores clase A}
\end{center}

\begin{center}
Cedano Gutierrez Nancy Noemi\\
\textbf{01-oct-2019}\\
\textbf{Universidad Politecnica de La Zona Metropolitana de Guadalajara}
\end{center}

\section{Amplificadores de Clase A}

Un amplificador de potencia en clase A funciona cuando la tension de polarizacion y la amplitud maxima de la señal de entrada poseen valores tales que hacen que la corriente de salida circule durante todo el periodo de la señal de entrada.
Esta clase de amplificacion tiene el inconveniente de generar una fuerte y constante emision de calor. Por ello, los transistores de salida estan a una temperatura fija y sin alteraciones. Es frecuente encontrar este en circuitos de audio y en los eauipos domesticos de gama alta, ya que proporciona una calidad de sonido potente y de muy buena calidad.
Los amplificadores de clase A consisten en un transistor de salida conectado al positivo de la fuente de alimentación y un transistor de corriente constante conectado de la salida al negativo de la fuente de alimentación. La señal del transistor de salida modula tanto el voltaje como la corriente de salida. Cuando no hay señal de entrada, la corriente de polarización constante fluye directamente del positivo de la fuente de alimentación al negativo, resultando que no hay corriente de salida, se gasta mucha corriente. Algunos amplificador de clase A más sofisticados tienen dos transistores de salida en configuración push-pull.
La clase A se refiere a una etapa de salida con una corriente de polarización mayor que la máxima corriente de salida que dan, de tal forma que los transistores de salida siempre están consumiendo corriente. La gran ventaja de la clase A es que es casi lineal, y en consecuencia la distorsión es menor.\\
\section{Como funciona}
El amplificador de clase A ofrece como ventaja, que la señal de salida no aparece distorsionada y como principal inconveniente, que su rendimiento máximo es de 25 por ciento y de un 50 por ciento cuando la carga es acoplada por un transformador.
Un amplificador de clase A presenta a su salida una señal copia de la de la entrada pero amplificada y sin distorsión.
En este caso la máxima señal de salida se obtendrá cuando el punto estático coincida con el centro de la recta de carga, consiguiendo, por tanto, la máxima potencia de salida.
Se pueden presentar dos casos, se han de ver bajo el punto de vista de las rectas de carga: E n el primero, son distintas para cc y para ca y en el segundo caso coinciden para cargas resistivas.
\end{document}