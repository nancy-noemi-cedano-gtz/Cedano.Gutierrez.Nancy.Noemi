\documentclass[11pt,a4paper]{article}
\usepackage{graphicx}
\usepackage{amsmath}
\usepackage{amssymb}
\usepackage{mathrsfs}
\usepackage{cancel}

\begin{document}
\begin{center}
\textbf{TAREA 5}\\
\title{MODULACION DE ANCHO DE PULSO CON AMP-OP Y TRANSISTORES}
DISEÑO DE UNA MODULACION DE ANCHO DE PULSO (PWM) CON AMP-OP Y TRANSISTORES 
\end{center}

\begin{center}
\textbf{Alumna:}\\
CEDANO GUTIERREZ NANCY NOEMI\\
22 OCTUBRE DEL 2019\\
Universidad Politecnica de La Zona Metropolitana de Guadalajara
\end{center}

\section{MODULACION DE ANCHO DE PULSO}
Una modulación de ancho de pulso (PWM) es una técnica que logra producir el efecto de una señal analógica sobre una carga, a partir de la variación de la frecuencia y ciclo de trabajo de una señal digital. 
El ciclo de trabajo describe la cantidad de tiempo que la señal está en un estado lógico alto, como un porcentaje del tiempo total que este toma para completar un ciclo completo. La frecuencia determina que tan rápido se completa un ciclo (por ejemplo: 1000 Hz corresponde a 1000 ciclos en un segundo), y por consiguiente que tan rápido se cambia entre los estados lógicos alto y bajo. Al cambiar una señal del estado alto a bajo a una tasa lo suficientemente rápida y con un cierto ciclo de trabajo, la salida parecerá comportarse como una señal analógica constante cuanto esta está siendo aplicada en algún dispositivo.\\
\section{APLICACIONES}
Las señales de PWM son utilizadas comunmente en el control de aplicaciones. Su uso principal es el control de motores de corriente continua, aunque también pueden ser utilizadas para controlar válvulas, bombas, sistemas hidráulicos, y algunos otros dispositivos mecánicos. La frecuencia a la cual la señal de PWM se generará, dependerá de la aplicación y del tiempo de respuesta del sistema que está siendo controlado.\\
La principal desventaja que presentan los circuitos PWM es la posibilidad de que haya interferencias generadas por radiofrecuencia. Éstas pueden minimizarse ubicando el controlador cerca de la carga y realizando un filtrado de la fuente de alimentación.\\
La construcción típica de un circuito PWM se lleva a cabo mediante un comparador con dos entradas y una salida. Una de las entradas se conecta a un oscilador de onda dientes de sierra, mientras que la otra queda disponible para la señal moduladora. En la salida la frecuencia es generalmente igual a la de la señal dientes de sierra y el ciclo de trabajo está en función de la portadora.\\

\end{document}